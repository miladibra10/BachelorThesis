\newpage
\thispagestyle{empty}
\begin{latin}
    \subsection*{Abstract}
    Due to the growth of online businesses and the need for stability and high efficiency for providing services for customers, new architectures should be used for software development. The main characteristics of these architectures are scalability and being fault-tolerant.

    Microservice, among these architectures, is one of the most popular ones. The focus of this architecture is on scalability, extendability and independence of different services from each other. The way this architecture achieves the mentioned features is through separating different functionalities into different services and deploying them in isolated environments.

    Similar to any architecture, the proper implementation of microservice architecture has several challenges. Due to the decentralized nature of the systems in this architecture, connecting users to the services, will not be as simple as in the past because the user needs to gather information from different parts of the system. Therefore, changing the way users communicate with the system and upgrading user applications will be expensive.

    API Gateway pattern is designed to solve this challenge. In this pattern, users still send their requests to only one median system. The task of this median system is to receive requests and redirect them to associated services.

    In spite of the existence of various open source and enterprise API Gateways, most of them lack extendability and configurability. The main purpose of this project is to implement an API Gateway that is extendable and configurable.

    \noindent {\bf Keywords:} API Gateway, Microservice architecture, Decentralized systems, Extendable systems.
\end{latin}
\cleardoublepage
\subsection*{پیاده‌سازی میان‌افزار خاص منظوره}
برای پیاده‌سازی یک میان‌افزار خاص منظوره، ابتدا باید از نصب بودن زبان برنامه نویسی Golang در دستگاه مطمئن شویم.

طبق تعریف میان‌افزار در بخش
\ref{subsubsec:impl_middleware}
، یک میان‌افزار یک struct است که تابع Process را پیاده سازی می‌کند. برای مثال می‌توان برای اضافه‌کردن یک سرتیتر به جواب، قطعه کد زیر را نوشت.

\begin{latin}
  \begin{lstlisting}
package main

import (
  "net/http"
)

type AddHeaderMiddleware struct {
  key string
  val string
}

func (a AddHeaderMiddleware) Process(handler http.HandlerFunc) http.HandlerFunc {
  return func(w http.ResponseWriter, r *http.Request) {
    w.Header().Set(a.key, a.val)
    handler(w, r)
  }
}

func New(params map[string]interface{}) (interface{}, error) {
  return AddHeaderMiddleware{
    key: params["key"].(string),
    val: params["val"].(string),
  }, nil
}

  \end{lstlisting}
\end{latin}

همچنین وجود تابع New با ورودی از نوع Map به عنوان تابع سازنده‌ی میان‌افزار الزامی است.

پس از پیاده‌سازی میان‌افزار مورد نظر، می‌توان آن‌ را جهت اضافه کردن به برنامه کامپایل کرد. دستور زیر برای انجام این کار مناسب است.

\begin{latin}
  \begin{lstlisting}
go build -buildmode=plugin <file_name>.go
  \end{lstlisting}
\end{latin}

این دستور یک فایل با فرمت so خواهد ساخت. مسیر این فایل را می‌توان در مشخصه‌ی \lr{middlewarePath}، استفاده کرد.
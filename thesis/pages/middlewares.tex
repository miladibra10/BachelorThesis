\subsection*{تنظیمات میان‌افزار‌ها}
طبق تعریف میان‌افزار‌ها در بخش
\ref{subsubsec:impl_middleware}
، مشخصه‌های یک میان‌افزار به شکل زیر تعیین می‌شود.


\subsubsection*{مشخصه‌ی middlewareName}
این مشخصه،‌ تعیین‌کننده‌ی نام میان‌افزار خواهد بود. در صورت عدم وجود این نام در مجموعه میان‌افزار‌های پیاده‌سازی شده‌ی داخلی، اجرای برنامه با خطا مواجه خواهد شد.

\subsubsection*{مشخصه‌ی refName}
مشخصه‌ی نام مرجع،‌ جهت استفاده در ضابطه‌ها قرار‌داده شده است. در صورت عدم وجود این مشخصه، middlewareName به عنوان نام مرجع انتخاب خواهد شد. از این مشخصه می‌توان برای ساخت چند عدد از یک نوع مشخص از میان‌افزار‌ها ولی با تنظیمات متفاوت، استفاده کرد.

\subsubsection*{مشخصه‌ی customMiddleware}
با توجه به گسترش‌پذیری سامانه، می‌توان یک میان‌افزار خارجی را در برنامه استفاده کرد. این مشخصه هنگام استفاده از میان‌افزار‌های خاص‌منظوره استفاده می‌شود. نوع این مشخصه boolean است.

\subsubsection*{مشخصه‌ی middlewarePath}
در صورت استفاده از میان‌افزار خارجی، مسیر فایل میان‌افزار مورد نظر با این مشخصه تعیین می‌شود.

\subsubsection*{مشخصه‌ی params}
در این مشخصه، تنظیمات خاص هر میان‌افزار تعیین می‌شود. این مشخصه از نوع Map است.

نمونه‌ی تنظیمات میان‌افزار‌ها به شکل زیر است:

\begin{latin}
  \begin{lstlisting}
middlewares:
  - middlewareName: prometheus
  - middlewareName: cors
    params:
      allowOrigins:
        - "*"
  - middlewareName: ratelimit
    params:
      strategy: ip
      limitValue: 10
      limitDuration: "1m"

  \end{lstlisting}
\end{latin}


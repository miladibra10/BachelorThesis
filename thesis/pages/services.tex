\subsection*{تنظیمات خدمت‌ها}
با توجه به تعریف خدمت در فصل
\ref{sec:implementation}
، سامانه می‌تواند درخواست‌ها را به چندین خدمت مورد نظر متصل کرد. در این بخش از تنظیمات، مجموعه‌ی خدمت‌ها به برنامه شناسانده می‌شود.

هر خدمت دارای سه مشخصه‌ی \lr{name}، \lr{loadbalancer} و \lr{backends} است.


\subsubsection*{مشخصه‌ی name}
این مشخصه‌، نام خدمت را تعیین می‌کند.

\subsubsection*{مشخصه‌ی loadbalancer}
این مشخصه‌، الگوریتم مورد استفاده جهت توازن بار بین بطن‌های خدمت‌ را مشخص می‌کند. در حال حاضر تنها الگوریتم random توسط این برنامه پشتیبانی می‌شود.

\subsubsection*{مشخصه‌ی backends}
مجموعه بطن‌های یک خدمت، توسط این مشخصه تعیین می‌شود. هر بطن دارای مشخصه‌های نام، وزن و آدرس است.

نمونه‌ی تنظیمات خدمت‌ها به شکل زیر است:

\begin{latin}
    \begin{lstlisting}
services:
    - name: test-service
      loadbalancer: random
      backends:
        - name: first
          weight: 1
          address: http://localhost:9092
        - name: second
          weight: 2
          address: http://localhost:9093
    \end{lstlisting}
\end{latin}


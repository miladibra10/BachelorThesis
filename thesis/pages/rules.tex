\subsection*{تنظیمات ضابطه‌ها}
برای تعیین تنظیمات مجموعه‌ی ضابطه‌ها باید از ۹ مشخصه‌ی زیر استفاده کرد.


\subsubsection*{مشخصه‌ی name}
این مشخصه، نام ضابطه را تعیین می‌کند.

\subsubsection*{مشخصه‌ی service}
این مشخصه، نام خدمتی است که درخواست‌های کاربران را طبق تنظیمات مسیر‌یابی ضابطه، به آن منتقل می‌شود. این خدمت باید در قسمت مربوط به خدمت‌ها، تعریف شده باشد.

\subsubsection*{مشخصه‌ی schema}
برای تعیین پروتکل ضابطه از این مشخصه استفاده می‌شود. در حال حاضر، تنها پروتکل‌های http و https توسط برنامه پشتیبانی می‌شوند.

\subsubsection*{مشخصه‌ی pathPrefix}
این مشخصه برای تعیین پیش‌مسیرهایی است که ضابطه بر روی آن اعمال می‌شود.

\subsubsection*{مشخصه‌ی hosts}
این مشخصه مجموعه مبدا‌ های ممکن برای درخواست ها را مشخص می‌کند. در صورت عدم وجود این مشخصه،‌ دسترسی سامانه از هیچ محدودیتی در مورد مبدا نخواهد داشت.

\subsubsection*{مشخصه‌ی methods}
این مشخصه مجموعه method های اعمال شده بر ضابطه، که در پروتکل http تعریف شده‌اند، را تعیین می‌کند.

\subsubsection*{مشخصه‌ی headers}
این مشخصه از نوع Map است. این Map شامل سرتیتر‌های درخواست‌ها و مقادیر سرتیتر‌ها خواهد بود. در صورت مطابقت سرتیتر‌های درخواست کاربر و این Map، درخواست به خدمت منتقل خواهد شد.

\subsubsection*{مشخصه‌ی queries}
این مشخصه از نوع Map است. این Map شامل مولفه‌های پرس‌وجوی درخواست‌ها و مقادیر آن‌ها خواهد بود. در صورت مطابقت مولفه‌های پرس‌و‌جوی درخواست کاربر و این Map، درخواست به خدمت منتقل خواهد شد.

\subsubsection*{مشخصه‌ی middlewares}
این مشخصه مجموعه‌ی میان‌افزار‌های اعمال شده بر روی ضابطه را تعیین می‌کند. نام مرجع میان‌افزار باید در قسمت تنظیمات میان‌افزار‌ها موجود باشد.

نمونه‌ی تنظیمات ضابطه‌ها به شکل زیر است:

\begin{latin}
  \begin{lstlisting}
rules:
  - service: test-service
    schema: http
    pathPrefix: /
    hosts:
      - www.example.com
    methods:
      - GET
      - POST
    headers:
      key: val
      key2: val2
    queries:
      key: val
      key2: val2
    middlewares:
      - prometheus
      - cors
      - ratelimit

  \end{lstlisting}
\end{latin}


\section{روش پیاده‌سازی}\label{sec:implementation}

\subsection{مقدمه}\label{subsec:impl_intro}
برای پیاده‌سازی یک سامانه، می‌توان دو رویکرد داشت. می‌توان از سامانه‌های از پیش موجود استفاده کرد و با حاصل کردن تغییرات لازم به مقصود رسید. و یا می‌توان با پیاده‌سازی سامانه از ابتدا، نیاز‌های را حل کرد.

از طرفی با توجه به ویژگی‌های ذکرشده در فصل
\ref{sec:sources}
، راه حل انتخابی برای پیاده‌سازی یک درگاه ارتباط با رابط‌های برنامه‌نویسی باید معیار‌های زیر را مدنظر خود قرار داده باشد:

\begin{itemize}
    \item سرعت بالا
    \item کارایی بالا
    \item بهینه‌بودن راه‌حل
    \item گسترش‌پذیری
    \item تغییر‌پذیری
    \item سادگی و قابل فهم بودن
\end{itemize}

\subsection{انتخاب روش و فناوری}\label{subsec:impl_choose}
استفاده‌ از فناوری‌های موجود، مانند Nginx، HAProxy و ... و ایجاد تغییرات در آن‌ها جهت بر طرف کردن نیاز‌های مطرح شده در فصل
\ref{sec:sources}
، می‌تواند یکی از راه‌حل ‌های حل مسئله باشد. ولی با توجه به پایه‌ی پیاده‌سازی این محصولات بر اساس نیاز‌های گذشته، و همچنین عدم سازگاری این محصولات با محیط‌های ابری، جهت استفاده به عنوان درگاه ارتباط با رابط‌های برنامه‌نویسی، مشکلات زیادی برای حل مسئله وجود خواهد آمد. برخی از این مشکلات عبارتند از:

\begin{itemize}
    \item اجبار به استفاده از راه‌حل‌های موقتی برای سازگاری این محصولات با فضای موجود
    \item پیچیدگی بیش از حد به دلیل وجود ویژگی‌های مرتبط با حوزه‌های دیگر
    \item احتمال عدم سازگاری تغییرات مورد نظر با ویژگی‌های اصلی محصول
    \item و \ldots
\end{itemize}

از این رو پیاده‌سازی یک درگاه، از ابتدا و با توجه به نیاز‌های جدید،‌ راه حل منطقی تری به نظر می‌رسد.

با توجه به نیاز سامانه‌ به دسترسی‌های سطح پایین، جهت ارتباط با لایه‌ی هفتم شبکه، و همچنین نیاز به سرعت بالا برای کنترل بار، زبان‌های برنامه‌نویسی محدودی مناسب پیاده‌سازی این محصول خوا‌هند بود. برخی از این گزینه‌ها عبارتند از:

\begin{itemize}
    \item زبان برنامه‌نویسی کامپایلری C/C++
    \item زبان برنامه‌نویسی کامپایلری Golang
    \item زبان برنامه‌نویسی مفسری Lua
    \item زبان برنامه‌نویسی Javascript (بستر \lr{Node.js})
\end{itemize}

معمولا زبان‌های برنامه‌نویسی سطح پایین‌تر، سرعت بالاتری نیز دارند. از این رو زبان‌های C و C++ از بالاترین سرعت برخوردار هستند. با این وجود، کاری‌هایی مانند مدیریت حافظه، مدیریت پردازه‌های هم‌روند و ... به عهده‌ی برنامه‌نویس است. این امر باعث پیچیدگی توسعه خواهد بود. از این رو، زبان برنامه‌نویسی Golang، که یک زبان برنامه‌نوسی سطح‌ پایین محسوب می‌شود، به علت مدیریت حافظه خودکار و استفاده‌ی ساده از پردازه‌ها و ریسمان‌ها برای اجرای فرآیند‌های هم‌روند، می‌تواند گزینه‌ی مناسبی جهت پیاده‌سازی باشد.

از زبان‌های مفسری نیز می‌توان برای توسعه‌ی نرم‌افزار‌هایی که در زمان اجرا نیاز به تغییر در خود دارند، استفاده کرد. ولی نقطه‌ی تاریک استفاده ‌از زبان‌های برنامه‌نویسی مفسری، احتمال بالای رخداد خطا‌های زمان اجرا خواهد بود. این ویژگی باعث از دست رفتن ثبات سامانه می‌شود.

نحوه‌ی انجام فرآیند‌های هم‌روند نیز در انتخاب فناوری پیاده‌سازی این بخش از سامانه بسیار تاثیرگذار است. زیرا با توجه به نیاز به توان عملیاتی بالا، فناوری‌های تک رشته‌ای و یا تک‌ پردازه‌ای باعث کاهش توان عملیاتی سامانه خواهند شد.

جدول
\ref{tab:choice}
شامل مقایسه‌ی زبان‌های برنامه‌نویسی ذکر شده با توجه‌ به معیار‌های ذکر شده است.

\begin{table}[H]
    \centering
    \caption{مقایسه‌ی زبان‌های برنامه‌نویسی جهت پیاده‌سازی درگاه ارتباط با رابط‌های برنامه‌نویسی}\label{tab:choice}
    \begin{tabular}{|c|c|c|c|c|c|c|}
        \hline
        زبان برنامه‌نویسی & سرعت & سادگی & بهینگی & توان عملیاتی & مدیریت حافظه & مدیریت فرآیند‌های هم‌روند\\
        \hline
        C/C++ & بسیار بالا & پایین & بسیار بالا & بسیار بالا & خیر & بسیار سخت\\
        \hline
        Golang & بالا & متوسط & بالا & بالا & بله & آسان\\
        \hline
        Lua & بالا & متوسط & بالا & متوسط & بله & سخت\\
        \hline
        Javascript & متوسط & بالا & پایین & متوسط & بله & سخت\\
        \hline
    \end{tabular}

\end{table}


با توجه به جدول
\ref{tab:choice}
، روش انتخابی برای حل مسئله،‌ پیاده‌سازی سامانه از ابتدا و با استفاده از زبان برنامه‌نویسی Golang است. زیرا با اینکه سرعت کمتری نسبت به زبان برنامه‌نویسی C یا ++C دارد، ولی با توجه به عدم نیاز به مدیریت حافظه، مدیریت آسان‌تر فرآیند‌های هم‌روند و همچنین سادگی و بهینگی قابل قبول،‌ می‌توان از اختلاف سرعت این دو زبان برنامه‌نویسی چشم پوشی کرد.

\subsection{پیاده‌سازی}\label{subsec:impl_impl}
متن‌ها


\cleardoublepage 
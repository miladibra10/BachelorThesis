\section{پیشنهاد‌ها}\label{sec:recom}
پیاده‌سازی یک راه‌کار جامع برای محیط‌های ابری، دارای شرایط و سختی‌های مخصوص به خود است. برخی از ویژگی‌ها برای استفاده‌ی صنعتی از این نوع سامانه‌ها ضروری است. برخی از این ویژگی‌ها که در نسخه‌ی پیاده‌سازی شده وجود ندارند، عبارتند از:

\begin{itemize}
    \item پنل و رابط برنامه‌نویسی مدیریت سامانه
    \item پشتیبانی از پروتکل‌های \lr{websocket}، \lr{gRPC} و\ldots
    \item برخی میان‌افزار‌های پرکاربرد
    \item یکپارچگی با برخی از سکو‌های ابری
\end{itemize}

\subsection{پنل و رابط برنامه‌نویسی مدیریت سامانه}\label{subsec:recom_panel}
وجود پنل و رابط‌های برنامه‌نویسی برای مدیریت پیکر‌بندی‌ها و عدم نیاز به راه‌اندازی مجدد سامانه، علیرغم اینکه برخی از نسخه‌های تجاری فاقد آن هستند، یکی از ویژگی‌های ضروری یک درگاه ارتباط با رابط‌های برنامه‌نویسی است. وجود این ویژگی باعث عدم وابستگی سازمان‌های استفاد کننده به افراد با تخصص بالا است.

\subsection{پنل و رابط برنامه‌نویسی مدیریت سامانه}\label{subsec:recom_protocols}
در نسخه‌ی پیاده‌سازی شده، تنها پروتکل‌های HTTP و HTTPS پشتیبانی می‌شوند. امکان پشتیبانی از برخی از پروتکل‌ها،‌مانند \lr{gRPC}، \lr{websocket}، \lr{HTTP2} و … برای بسیاری از سامانه‌ها ضروری هستند.
ترجمه‌ی پروتکل‌ها به یکدیگر نیز یکی از ویژگی‌های جذاب و پرکاربرد در درگاه‌های ارتباط محسوب می‌شود.

\subsection{میان‌افزار‌های پرکاربرد}\label{subsec:recom_middlewares}
برخی از میان‌افزار‌های پرکاربرد، هزینه‌ی پیاده‌سازی بخش‌های مختلف سامانه را کاهش می‌دهند. وجود این میان‌افزار‌ها، باعث می‌شود استفاده از درگاه‌های ارتباط، از نظر زمانی و اقتصادی به‌صرفه‌ و منطقی باشند. برخی از این میان‌افزار‌ها عبارتند از:

\begin{itemize}
    \item میان‌افزار Caching
    \item میان‌افزار احراز هویت \LTRfootnote{Authentication} و کنترل دسترسی \LTRfootnote{Authorization}
    \item میان‌افزار‌های مربوط به امنیت سامانه
    \item میان‌افزار‌های ترکیب پاسخ‌های خدمت‌های مختلف
    \item میان‌افزار‌های تغییر درخواست و پاسخ درخواست
\end{itemize}

با توجه به معماری گسترش‌پذیر سامانه‌ی پیاده‌سازی شده، اضافه‌کردن میان‌افزار‌ها، به سادگی امکان‌پذیراست. با تحلیل نیاز‌های واقعی در صنعت و استخراج نیاز‌های هر کدام از میان‌افزار‌ها،‌ می‌توان این میان‌افزار‌ها را به طور پیش‌فرض در سامانه قرار داد.

\subsection{یکپارچگی با برخی از سکو‌های ابری}\label{subsec:recom_platforms}
با توجه به رشد روز‌افزون محیط‌های ابری و استفاده‌ی روزمره‌ی آن‌ها توسط مشتریان مختلف، امکان یکپارچه‌سازی با سکو‌های ابری معروف، یک ویژگی بسیار مهم برای درگاه‌های ارتباط خواهد بود. برخی از سکو‌های معروف خدمات ابری عبارتند از:

\begin{latin}
    \begin{itemize}
        \item Docker
        \item Kubernetes
        \item Openshift
        \item Openstack
        \item Marathon
        \item Mesos
    \end{itemize}
\end{latin}

با توجه به معماری گسترش‌پذیر بخش تامین‌کنندگان، که در حال حاضر از تامین‌کننده‌ی فایل پشتیبانی می‌کند، می‌توان با کمی مطالعه‌ی رابط‌های برنامه‌نویسی سکو‌های ذکر شده، تامین‌کنندگان جدیدی به سامانه اضافه کرد.


\cleardoublepage 
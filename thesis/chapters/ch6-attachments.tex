\section{پیوست‌ها}\label{sec:attachments}

\section*{واژه‌نامه فارسی به انگلیسی}\label{sec:gloss}
\addcontentsline{toc}{section}{\numberline{} واژه‌نامه فارسی به انگلیسی}
\thispagestyle{empty}

\englishgloss{Microservice}{میکرو سرویس}
\englishgloss{Monolithic}{یک‌پارچه}
\englishgloss{Centralized}{متمرکز}
\englishgloss{System}{سامانه}
\englishgloss{Interface}{رابط}
\englishgloss{API (Application Programming Interface)}{رابط‌ برنامه نویسی}
\englishgloss{API Gateway}{درگاه ارتباط با رابط‌های برنامه‌نویسی}
\englishgloss{Routing}{مسیردهی}
\englishgloss{Router}{مسیریاب}
\englishgloss{Scalability}{مقیاس‌مندی}
\englishgloss{Extendability}{گسترش پذیری}
\englishgloss{Load Balancing}{توازن بار}
\englishgloss{Monitoring}{رصد}
\englishgloss{Single Point of Failure}{تک نقطه‌ی شکست}
\englishgloss{Platform}{سکو}
\englishgloss{Process}{پردازه}
\englishgloss{Thread}{ریسمان}
\englishgloss{Package Oriented}{بسته‌مبنا}
\englishgloss{Backend}{بطن}
\englishgloss{Service}{خدمت}
\englishgloss{Load Balancer}{متعادل‌کننده‌ی بار}
\englishgloss{Rule}{ضابطه}
\englishgloss{Middleware}{میان‌افزار}
\englishgloss{Service Provider}{تامین‌کننده‌ی خدمت}
\englishgloss{Unit Test}{آزمون واحد}
\englishgloss{Authentication}{احراز هویت}
\englishgloss{Authorization}{کنترل دسترسی}

\cleardoublepage

\section*{راهنمای کاربران}\label{subsec:manual}
\subsection*{راهنمای نصب}
برای نصب درگاه ارتباط با رابط‌های برنامه‌نویسی پیاده‌سازی شده، می‌توان از دو روش زیر بهره گرفت:

\begin{itemize}
    \item استفاده از سکوی Docker
    \item کامپایل و نصب برنامه
\end{itemize}

برای استفاده از روش‌های ذکر شده، ابتدا مخزن برنامه را با استفاده از دستور زیر بارگیری کنید.

\begin{latin}
    \begin{lstlisting}
git clone https://github.com/Highway-Project/highway
    \end{lstlisting}
\end{latin}


\subsubsection*{استفاده از سکوی Docker}
برای نصب برنامه با این روش، ابتدا از نصب بودن Docker در دستگاه خود اطمینان حاصل کنید. سپس در محیط خط فرمان، به پوشه‌ی برنامه رفته و دستور زیر را اجرا کنید:

\begin{latin}
    \begin{lstlisting}
make docker-build
    \end{lstlisting}
\end{latin}


\subsubsection*{کامپایل و نصب برنامه}
این روش برای اجرا در محیط سیستم‌عامل لینوکس مناسب است.

برای نصب برنامه با این روش، ابتدا از نصب بودن زبان برنامه‌نویسی Golang در دستگاه خود اطمینان حاصل کنید. سپس برنامه را با استفاده از دستور زیر در محیط خط فرمان، ‌کامپایل کنید.


\begin{latin}
    \begin{lstlisting}
make build
    \end{lstlisting}
\end{latin}

پس از اجرای دستور فوق، فایل قابل اجرای این برنامه در پوشه‌ی bin و با نام highway موجود خواهد بود.

\subsection*{راهنما‌ی تنظیمات}
تنظیمات برنامه در قالب یک فایل با فرمت yaml تعریف می‌شوند. این تنظیمات شامل ۵ قسمت است. قسمت‌های مختلف تنظیمات عبارتند از:

\begin{itemize}
    \item تنظیمات سراسری \LTRfootnote{Global}
    \item تنظیمات مسیریاب
    \item تنظشمات خدمت‌ها
    \item تنظیمات میان‌افزار‌ها
    \item تنظیمات ضابطه‌ها
\end{itemize}

\subsection*{تنظیمات سراسری}
رفتار کلی سامانه با استفاده از این تنظیمات تعیین می‌شود. این تنظیمات دارای مشخصه‌های زیر است:

\subsubsection*{مشخصه‌ی port}
این مشخصه، port مورد استفاده برای برنامه، جهت اجرای برنامه را تعیین می‌کند

\subsubsection*{مشخصه‌ی readHeaderTimeout}
میزان زمان قابل قبول جهت خواندن سرتیتر‌های درخواست‌های کاربران، با استفاده از این مشخصه معین می‌شود. واحد این مشخصه میلی‌ثانیه است.

استفاده از این مشخصه برای مقابله با برخی از حمله‌هایی که حمله‌کننده با ارسال کند اطلاعات، منابع سامانه را مشغول و از دسترس خارج می‌کند، مناسب است.

\subsubsection*{مشخصه‌ی readTimeout}
میزان زمان قابل قبول جهت خواندن درخواست‌های کاربران، با استفاده از این مشخصه معین می‌شود. واحد این مشخصه میلی‌ثانیه است.

استفاده از این مشخصه برای مقابله با برخی از حمله‌هایی که حمله‌کننده با ارسال کند اطلاعات، منابع سامانه را مشغول و از دسترس خارج می‌کند، مناسب است.

\subsubsection*{مشخصه‌ی writeTimeout}
میزان زمان قابل قبول جهت پاسخ دادن به درخواست‌های کاربران، با استفاده از این مشخصه معیین می‌شود. واحد این مشخصه میلی‌ثانیه است. این مشخصه برای مدیریت پنجره‌های دریافت در ارتباط‌های TCP مناسب است.

\subsubsection*{مشخصه‌ی idleTimeout}
این مشخصه، تعیین کننده‌ی میزان زمان باز ماندن ارتباط بین کاربر و سامانه در صورت عدم استفاده از راه ارتباطی است. واحد این مشخصه میلی‌ثانیه است. با استفاده از این مشخصه، می‌توان تعداد ارتباط‌های بی استفاده را، جهت مدیریت بهتر منابع،‌ مدیریت کرد.

\subsubsection*{مشخصه‌ی maxHeaderBytes}
یشترین حجم سرتیتر‌های درخواست با این مشخصه قابل تعیین است. یکی از کاربرد‌های این مشخصه، جلوگیری از ارسال درخواست‌های با حجم بالا است.

نمونه‌ی یک تنظیمات سراسری، در زیر آمده است:

\begin{latin}
    \begin{lstlisting}
global:
    port: 8080
    readTimeout: 10000
    readHeaderTimeout: 10000
    writeTimeout: 10000
    idleTimeout: 10000
    maxHeaderBytes: 8196
    \end{lstlisting}
\end{latin}


\subsection*{تنظیمات مسیریاب}
کتابخانه‌های مختلفی را می‌توان جهت مسیریابی درخواست‌ها در زبان برنامه‌نویسی Golang استفاده کرد. یکی از این کتابخانه‌ها که توسط برنامه پشتیبانی می‌شود، کتابخانه‌ی gorilla/mux است. در این قسمت از تنظیمات، باید نام کتابخانه‌ی مورد استفاده جهت مسیریابی درخواست‌ها ذکر شود.

نمونه‌ی این تنظیمات به شکل زیر است:

\begin{latin}
    \begin{lstlisting}
router:
    name: gorilla
    \end{lstlisting}
\end{latin}

\subsection*{تنظیمات خدمت‌ها}
با توجه به تعریف خدمت در فصل
\ref{sec:implementation}
، سامانه می‌تواند درخواست‌ها را به چندین خدمت مورد نظر متصل کرد. در این بخش از تنظیمات، مجموعه‌ی خدمت‌ها به برنامه شناسانده می‌شود.

هر خدمت دارای سه مشخصه‌ی \lr{name}، \lr{loadbalancer} و \lr{backends} است.


\subsubsection*{مشخصه‌ی name}
این مشخصه‌، نام خدمت را تعیین می‌کند.

\subsubsection*{مشخصه‌ی loadbalancer}
این مشخصه‌، الگوریتم مورد استفاده جهت توازن بار بین بطن‌های خدمت‌ را مشخص می‌کند. در حال حاضر تنها الگوریتم random توسط این برنامه پشتیبانی می‌شود.

\subsubsection*{مشخصه‌ی backends}
مجموعه بطن‌های یک خدمت، توسط این مشخصه تعیین می‌شود. هر بطن دارای مشخصه‌های نام، وزن و آدرس است.

نمونه‌ی تنظیمات خدمت‌ها به شکل زیر است:

\begin{latin}
    \begin{lstlisting}
services:
    - name: test-service
      loadbalancer: random
      backends:
        - name: first
          weight: 1
          address: http://localhost:9092
        - name: second
          weight: 2
          address: http://localhost:9093
    \end{lstlisting}
\end{latin}


\subsection*{تنظیمات میان‌افزار‌ها}
طبق تعریف میان‌افزار‌ها در بخش
\ref{subsubsec:impl_middleware}
، مشخصه‌های یک میان‌افزار به شکل زیر تعیین می‌شود.


\subsubsection*{مشخصه‌ی middlewareName}
این مشخصه،‌ تعیین‌کننده‌ی نام میان‌افزار خواهد بود. در صورت عدم وجود این نام در مجموعه میان‌افزار‌های پیاده‌سازی شده‌ی داخلی، اجرای برنامه با خطا مواجه خواهد شد.

\subsubsection*{مشخصه‌ی refName}
مشخصه‌ی نام مرجع،‌ جهت استفاده در ضابطه‌ها قرار‌داده شده است. در صورت عدم وجود این مشخصه، middlewareName به عنوان نام مرجع انتخاب خواهد شد. از این مشخصه می‌توان برای ساخت چند عدد از یک نوع مشخص از میان‌افزار‌ها ولی با تنظیمات متفاوت، استفاده کرد.

\subsubsection*{مشخصه‌ی customMiddleware}
با توجه به گسترش‌پذیری سامانه، می‌توان یک میان‌افزار خارجی را در برنامه استفاده کرد. این مشخصه هنگام استفاده از میان‌افزار‌های خاص‌منظوره استفاده می‌شود. نوع این مشخصه boolean است.

\subsubsection*{مشخصه‌ی middlewarePath}
در صورت استفاده از میان‌افزار خارجی، مسیر فایل میان‌افزار مورد نظر با این مشخصه تعیین می‌شود.

\subsubsection*{مشخصه‌ی params}
در این مشخصه، تنظیمات خاص هر میان‌افزار تعیین می‌شود. این مشخصه از نوع Map است.

نمونه‌ی تنظیمات میان‌افزار‌ها به شکل زیر است:

\begin{latin}
  \begin{lstlisting}
middlewares:
  - middlewareName: prometheus
  - middlewareName: cors
    params:
      allowOrigins:
        - "*"
  - middlewareName: ratelimit
    params:
      strategy: ip
      limitValue: 10
      limitDuration: "1m"

  \end{lstlisting}
\end{latin}


\subsection*{تنظیمات ضابطه‌ها}
برای تعیین تنظیمات مجموعه‌ی ضابطه‌ها باید از ۹ مشخصه‌ی زیر استفاده کرد.


\subsubsection*{مشخصه‌ی name}
این مشخصه، نام ضابطه را تعیین می‌کند.

\subsubsection*{مشخصه‌ی service}
این مشخصه، نام خدمتی است که درخواست‌های کاربران را طبق تنظیمات مسیر‌یابی ضابطه، به آن منتقل می‌شود. این خدمت باید در قسمت مربوط به خدمت‌ها، تعریف شده باشد.

\subsubsection*{مشخصه‌ی schema}
برای تعیین پروتکل ضابطه از این مشخصه استفاده می‌شود. در حال حاضر، تنها پروتکل‌های http و https توسط برنامه پشتیبانی می‌شوند.

\subsubsection*{مشخصه‌ی pathPrefix}
این مشخصه برای تعیین پیش‌مسیرهایی است که ضابطه بر روی آن اعمال می‌شود.

\subsubsection*{مشخصه‌ی hosts}
این مشخصه مجموعه مبدا‌ های ممکن برای درخواست ها را مشخص می‌کند. در صورت عدم وجود این مشخصه،‌ دسترسی سامانه از هیچ محدودیتی در مورد مبدا نخواهد داشت.

\subsubsection*{مشخصه‌ی methods}
این مشخصه مجموعه method های اعمال شده بر ضابطه، که در پروتکل http تعریف شده‌اند، را تعیین می‌کند.

\subsubsection*{مشخصه‌ی headers}
این مشخصه از نوع Map است. این Map شامل سرتیتر‌های درخواست‌ها و مقادیر سرتیتر‌ها خواهد بود. در صورت مطابقت سرتیتر‌های درخواست کاربر و این Map، درخواست به خدمت منتقل خواهد شد.

\subsubsection*{مشخصه‌ی queries}
این مشخصه از نوع Map است. این Map شامل مولفه‌های پرس‌وجوی درخواست‌ها و مقادیر آن‌ها خواهد بود. در صورت مطابقت مولفه‌های پرس‌و‌جوی درخواست کاربر و این Map، درخواست به خدمت منتقل خواهد شد.

\subsubsection*{مشخصه‌ی middlewares}
این مشخصه مجموعه‌ی میان‌افزار‌های اعمال شده بر روی ضابطه را تعیین می‌کند. نام مرجع میان‌افزار باید در قسمت تنظیمات میان‌افزار‌ها موجود باشد.

نمونه‌ی تنظیمات ضابطه‌ها به شکل زیر است:

\begin{latin}
  \begin{lstlisting}
rules:
  - service: test-service
    schema: http
    pathPrefix: /
    hosts:
      - www.example.com
    methods:
      - GET
      - POST
    headers:
      key: val
      key2: val2
    queries:
      key: val
      key2: val2
    middlewares:
      - prometheus
      - cors
      - ratelimit

  \end{lstlisting}
\end{latin}



\subsection*{راهنمای اجرا}
با توجه به دو روش نصب برنامه، روش‌های اجرا نیز متفاوت خواهد بود.

\subsubsection*{اجرا روی سکوی Docker}
پس از نصب برنامه با استفاده از Docker و آماده سازی فایل تنظیمات، می‌توان برنامه را با دستور زیر اجرا کرد:

\begin{latin}
  \begin{lstlisting}
docker run -p <port>:<port> -v <path_to_config_file>:/opt/highway/config.yml highwayproject/highway
  \end{lstlisting}
\end{latin}

\textbf{نکته:}
مقدار port باید همان port تعیین شده در فایل تنظیمات باشد.

\subsubsection*{اجرای برنامه‌ی کامپایل شده}
پس از کامپایل برنامه و ایجاد شدن فایل قابل اجرا در پوشه‌ی bin، می‌توان برنامه را با دستور زیر اجرا کرد.

\begin{latin}
  \begin{lstlisting}
./bin/highway
  \end{lstlisting}
\end{latin}

\textbf{نکته:}
نکته: جهت اجرای درست برنامه، فایل تنظیمات باید در مسیر زیر موجود باشد:

\begin{latin}
  \begin{lstlisting}
/opt/highway/config.yml
  \end{lstlisting}
\end{latin}

\subsection*{پیاده‌سازی میان‌افزار خاص منظوره}
برای پیاده‌سازی یک میان‌افزار خاص منظوره، ابتدا باید از نصب بودن زبان برنامه نویسی Golang در دستگاه مطمئن شویم.

طبق تعریف میان‌افزار در بخش
\ref{subsubsec:impl_middleware}
، یک میان‌افزار یک struct است که تابع Process را پیاده سازی می‌کند. برای مثال می‌توان برای اضافه‌کردن یک سرتیتر به جواب، قطعه کد زیر را نوشت.

\begin{latin}
  \begin{lstlisting}
package main

import (
  "net/http"
)

type AddHeaderMiddleware struct {
  key string
  val string
}

func (a AddHeaderMiddleware) Process(handler http.HandlerFunc) http.HandlerFunc {
  return func(w http.ResponseWriter, r *http.Request) {
    w.Header().Set(a.key, a.val)
    handler(w, r)
  }
}

func New(params map[string]interface{}) (interface{}, error) {
  return AddHeaderMiddleware{
    key: params["key"].(string),
    val: params["val"].(string),
  }, nil
}

  \end{lstlisting}
\end{latin}

همچنین وجود تابع New با ورودی از نوع Map به عنوان تابع سازنده‌ی میان‌افزار الزامی است.

پس از پیاده‌سازی میان‌افزار مورد نظر، می‌توان آن‌ را جهت اضافه کردن به برنامه کامپایل کرد. دستور زیر برای انجام این کار مناسب است.

\begin{latin}
  \begin{lstlisting}
go build -buildmode=plugin <file_name>.go
  \end{lstlisting}
\end{latin}

این دستور یک فایل با فرمت so خواهد ساخت. مسیر این فایل را می‌توان در مشخصه‌ی \lr{middlewarePath}، استفاده کرد.


\cleardoublepage



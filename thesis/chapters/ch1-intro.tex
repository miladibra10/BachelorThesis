\section{مقدمه}\label{sec:subject}

\subsection{مقدمه}\label{subsec:subject-subject}


امروزه نمی‌توان نیاز به پنهان‌سازی داده‌ها را انکار کرد. این مسئله از جهاتی همچون مسائل امنیتی کاربردهای فراوانی دارد. آن‌چه که مهم است، میزان «امن» بودن روش‌های پنهان‌نگاری داده‌ها در یک مجموعه‌ی نگاه‌دارنده(که در اصطلاح «رسانه‌ی پوششی» خوانده می‌شود) است. یعنی نتوان با دیدن تصویری که حاوی اطلاعات مخفی است، به وجود داده‌ی مشکوکی در آن پی‌برد. به پنهان‌نگاری در اصطلاح لاتین، «استگانوگرافی»\footnote{\lr{Steganography}}
 گفته می‌شود.\\
استگانوگرافی در لغت(همان‌طور که ذکر شد) به معنای پنهان‌نگاری، یا نهان‌نگاری است و در اصطلاح به هنر نوشتن پیام مخفی اطلاق می‌شود، به گونه‌ای که کسی به‌جز فرستنده و گیرنده‌ی اصلی از وجود آن باخبر نشوند.\cite{Ming}\\
حال، برای آن‌که با روش‌های استگانوگرافی در تصویر آشنا شویم، ابتدا بررسی می‌کنیم که یک رایانه چه درکی از یک تصویر دارد. سپس می‌بینیم که پنهان‌نگاری در تصویر چگونه انجام می‌گیرد.

\cleardoublepage 